\documentclass{article}
\usepackage{amsmath}
\newtheorem{inferencerule}{Rule}[section]
\newtheorem{inferencecorollary}{Corollary}[inferencerule]
\usepackage{hyperref}
\hypersetup {
    colorlinks=true,
    urlcolor=blue,
    linkcolor=blue,
    filecolor=blue,
    citecolor=magenta
}
\usepackage{minted}
\begin{document}
\bibliographystyle{ieeetr}

\title{OVERVIEW OF DISAMBIGUATING TYPES IN JAVA PARTIAL PROGRAMS}

\author{Yong Qi Foo}

\maketitle            

\noindent \textbf{Summary.} The insights presented on 5 Feb 2021 are detailed here.
A few basic but widely-used inference rules for resolving type requirements are shown here.
A primitive and somewhat clumsy (but
working) implementation has been developed to resolve type requirements
based on those basic rules. That implementation is posted on
\href{https://github.com/yonggqiii/javacpp.git}{GitHub}.
The implementation still lacks many features
that will be implemented in the future.
More knowledge on how \cite{dagenais08} and other works disambiguate types need
to be acquired. The Java Language Specification for JDK11 needs to be studied.


%
\section{BACKGROUND}
%
The goal here is to be able to disambiguate Java Partial Programs (PPs) to enable their
compilation into Intermediate Representations (IRs) for program analysis. As \cite{dagenais08}
pointed out, while complete and sound disambiguation of PPs is an undecidable problem, we
can often trade correctness for increase precision.\\

\cite{dagenais08} has an implementation that fully disambiguates classes in Java 1.4 (and is
now deprecated),
and JCoffee \cite{jcoffee20} has and implementation that diambiguates classes in Java 8, but are unable
to disambiguate inner classes, generics, classes / functions as parameters, and lambda expressions.
The goal of this entire project is to build on these works to create a program that can 
perform disambiguation for any Java PP for Java 11.\\

This is a rather complex problem, and several issues must be solved for the program to work,
such as
\begin{enumerate}
    \item Missing type/attribute/method declarations
    \item Exception is never thrown
\end{enumerate}

For now, we will focus on point 1.

\section{APPROACH}
Several insights surface on analysis of JCoffee \cite{jcoffee20}.
\begin{enumerate}
    \item Compiler terminates before all possible errors are shown, hence the need for 
            more than one pass.
    \item Most (if not all) possible errors can be understood from the source code (such
            as by doing partial type inference in \cite{dagenais08})
    \item \texttt{UNKNOWN} class generated from JCoffee clutters programs and produces
            suboptimal structures during static analysis.
\end{enumerate}

Thus, our approach to resolve missing type declarations will instead model
itself after the partial type inference algorithm detailed in \cite{dagenais08}.
Our approach comprises four steps:
\begin{enumerate}
    \item Some Java compiler front-end (like Polyglot) to obtain a typed Abstract Syntax Tree (AST)
    \item From the typed AST, collect type information, consisting of
            (1) type definitions and (2) type requirements.
    \item Resolve type requirements
    \item Build surrounding stubs to complete the PP
\end{enumerate}

Points 2 and 3 will be elaborated here.

\section{TYPE DEFINITIONS}
Information on what is defined needs to be obtained so that our PP compiler can 
determine what constructs have already been defined. As such, we will
collect information on what we call Type Definitions.
These are unambiguous constructs that have been defined in the PP. 
They consist of fully-defined types, fully-defined attributes, and fully-defined
methods.
\subsection{FULLY-DEFINED TYPES}
These are classes / abstract classes / interfaces that have been declared (and consequently, defined),
containing fully-defined attributes and/or methods and their implementations (for concrete classes).
For example:
\begin{minted}{java}
class Foo {
    int hello() { return 1; }
}
class Bar {
    A a;
}
\end{minted}
Foo and Bar are fully-defined types.
\subsection{FULLY-DEFINED ATTRIBUTES}
These are attributes that have been declared in fully-defined types, where the names 
and types are given. For example:

\begin{minted}{java}
class Foo {
    Foo f;
}
class Bar {
    A a;
}
\end{minted}

\noindent show that \mintinline{java}{Foo.f} is a fully-defined attribute because
it is contained in \mintinline{java}{Foo},
which is fully-defined. The type of \mintinline{java}{Foo.f} also happens to be
fully-defined. \mintinline{java}{Bar.a} is a fully-defined attribute because it
is contained in \mintinline{java}{Bar},
which is fully-defined. Even though the type of \mintinline{java}{Bar.a} is not fully-defined,
it is referenced (its type has a name).

\subsection{FULLY-DEFINED METHODS}
These are the methods contained in fully-defined types,
where the exact method signatures of all variants are provided.
The types in the method signatures are all fully-defined, or referenced.
For example, in:
\begin{minted}{java}
class Foo {
    int hello() { return i; }
}
class Bar extends C {
    A meow(B b) { return b.woof() } 
}
\end{minted}

\noindent\mintinline{java}{Foo.hello} is a fully-defined method because it is contained
in a fully-defined type. The types in the method signature happen to also be fully-defined.
\mintinline{java}{Bar.meow} is also a fully-defined method as it is contained in a
fully-defined type. Even though the types in the signature of \mintinline{java}{Bar.meow}
are not fully-defined, they are referenced by an unambiguous name.

\section{TYPE REQUIREMENTS}
We also need to collect information of the types that are required for the PP to work,
but have not been fully defined by the PP. These
are what we call Type Requirements, which are constructs that have certain ambiguity
to their behaviour, and
must be added to the PP for the PP to be compilable.
They consist of referenced types, unknown types, referenced attributes and referenced
methods.

\subsection{REFERENCED TYPES}
These are types that have been referenced by a name somewhere in fully-defined types,
but are not be fully-defined. These types must exist with its exact given name,
but the specifics (attributes and methods and their corresponding types) are not given
by the PP.
For example, in:
\begin{minted}{java}
class Bar {
    A a;
}
\end{minted}
\noindent\mintinline{java}{A} is a referenced type because it is referenced by
\mintinline{java}{Bar} (and therefore must exist), but \mintinline{java}{A}
is not defined anywhere.

\subsection{UNKNOWN TYPES}
These are completely ambiguous types as they are never referenced\textemdash these
types might actually resolve to a fully-defined or referenced type,
or may be a completely separate type that was simply not referenced in the PP.
For example, in

\begin{minted}{java}
class Bar {
    A a;
    static void main(String[] args) {
        System.out.println(a.myCoolAttribute.anotherCoolAttribute);
    }
}
\end{minted}

\noindent\mintinline{java}{a.myCoolAttribute} and
\mintinline{java}{a.myCoolAttribute.anotherCoolAttribute} both resolve to unknown
types because they are not declared with any type information anywhere.

\subsection{REFERENCED ATTRIBUTES}
These are attributes of referenced / unknown types. These attributes must exist
in those types, but their types are unknown.
For example, in
\begin{minted}{java}
class Bar {
    A a;
    static void main(String[] args) {
        System.out.println(a.myCoolAttribute);
    }
}
\end{minted}

\noindent\mintinline{java}{A.myCoolAttribute} is an attribute that must exist,
but is of some unknown type.

\subsection{REFERENCED METHODS}
These are methods of referenced / unknown types.
These methods must exist in those types, but the method signatures are unknown.
For example, in

\begin{minted}{java}
class Bar {
    A a;
    static void main(String[] args) {
        System.out.println(a.myMethod(1));
    }
}
\end{minted}
\noindent\mintinline{java}{A.myMethod} is a referenced method,
where the method signature is unknown.

\section{RESOLVING TYPE REQUIREMENTS}
Once type information has been collected, the type requirements can be resolved into
the types, attributes and methods that will be built for the PP to work. These requirements
can be resolved by using information on the operations done on those types. These operations
operate on certain rules, which we will use to determine the behaviour of referenced or 
unknown types. For now, there are three broad categories of rules, being assignment,
attribute, and method call rules.

\begin{inferencerule}[General Assignment]
For some assignment statement \mintinline{java}{x = a;} where $type(x) = X$
and $type(a) = A$, then $X:>A$.
\end{inferencerule}
\begin{inferencecorollary}
If $X$ is final (non-extendable and by extension, fully-defined) then for the
program to be disambiguated, it must be possible for $X=A$. Thus, if 
$A$ is unknown, then $X=A$. Otherwise, the program cannot be disambiguated.\\
\end{inferencecorollary}

\noindent Suppose we have the following program:

\begin{minted}{java}
class Foo {
    A a;
    public static void main(String[] args) {
        String s = a.myAttribute; // Line 1
        String s = a;             // Line 2
    }
}
\end{minted}

\noindent Because \mintinline{java}{java.lang.String} is final, it is clear that
Line 1 resolves the type of \mintinline{java}{a.myAttribute} to be a \mintinline{java}{String},
therefore, \mintinline{java}{class A} must look something like:
\begin{minted}{java}
class A {
    String myAttribute;
    // ...
}
\end{minted}
However, Line 2 is impossible to disambiguate,
since \mintinline{java}{class A} cannot be the \mintinline{java}{String} type
and \mintinline{java}{class A} cannot extend \mintinline{java}{String}.
\begin{inferencecorollary}
If $A$ is fully-defined, for the program to be disambiguated, there must exist some
direct super-type of $A$ that is covariant to $X$.\\
\end{inferencecorollary}
\noindent Suppose we have the following program:
\begin{minted}{java}
class Foo extends B, implements C {
    public static void main(String[] args) {
        A a = new Foo();
    }
}
\end{minted}

\noindent For the program to be compilable, $B<:A$ or $C<:A$, which are both possible. However,
further suppose we have a new program

\begin{minted}{java}
class Foo {
    public static void main(String[] args) {
        A a = new Foo();
    }
}
\end{minted}
\noindent In this case, this program cannot be disambiguated as class Foo only
extends the Object class, and it has been defined that neither the Object class nor the Foo
class extend A.

\begin{inferencerule}[Variable Referencing]
    For some expression \mintinline{java}{x.y}, $y$ must be an attribute of $x$.
\end{inferencerule}

\noindent In general, even though we can be certain that $y$ is an attribute in $x.y$, we cannot
be certain about $x$. Because the following statement $a.b.c$ could mean `get the attribute c
in the attribute b of instance a of class A', or `get the attribute c of static inner class b
of class a'. Temporarily, we assume that there are no inner classes in our PPs, so we take 
it to be unambiguous that in $a.b.c$, a has attribute b, which has attribute c.

\begin{inferencecorollary}
    If $x$ is not declared as a local variable or attribute in the scope of reference,
    then $x$ itself must be a class or interface, so $y$ must be static.
\end{inferencecorollary}

\begin{inferencecorollary}
    if $x$ is an instance of some type $X$, then $X$ must not be an interface, as interface
    cannot have attributes.
\end{inferencecorollary}

\begin{inferencecorollary}
    For some expression \mintinline{java}{x.y = a}, 
    $type(x)$ cannot be an interface as interfaces cannot have instance attributes,
    and static attributes are final in interfaces. 
    Of course, as with Rule 1, $type(x.y) :> type(a)$
\end{inferencecorollary}

\begin{inferencerule}[Method calling]
    For some x.m(y), there must exist some method in x that accepts any type contravariant to y.
\end{inferencerule}

\begin{inferencecorollary}
    If $x$ is of some DefinedType where
\end{inferencecorollary}

\[
x.m =
\begin{cases}
    i_1 \rightarrow r_1\\
    i_2 \rightarrow r_2\\
    \vdots ,\\
    i_n \rightarrow r_n
\end{cases}
\enspace .
\]

Then $\exists k, type(y) <: i_k, 1 \leq k \leq n$

\begin{inferencecorollary}
    If $type(x)$ is referenced or unknown, then for some method call $x.m(y)$, adding
    a new overloaded method $m::type(y) \rightarrow \textrm{Unknown}$ would be sound.
\end{inferencecorollary}

\begin{inferencerule}[Constructors]
    If there exists a constructor for type $X$, then $X$ is a concrete class.
\end{inferencerule}

Overall, there exists many more inference rules, which will eventually be obtained from
the Java Language Specification.

\section{A PRIMITIVE TYPE RESOLVER}
Based on these few rules, very basic partial programs can be disambiguated.
Referring to the program developed \href{https://github.com/yonggqiii/javacpp.git}{here}
called JavaCPP (Java Compiler for Partial Programs) (just a temporary name),
the type resolving module can resolve simple partial programs, given the type
definitions and requirements. Two examples and their respective outputs from JavaCPP are
shown below.\\

Example 1.1 input PP.

\begin{minted}{java}
class Foo {
    A a;
    public static void main(String[] args) {
        Foo f = new Foo();
        f.a = new B();
        String s = f.a.doX(1).doY(false);
    }
}
\end{minted}

Example 1.2 output stubs.

\begin{minted}{java}
interface A {
    UnknownType1 doX(int i);
}
class B implements A { }
interface UnknownType1 {
    String doY(boolean b);
}
\end{minted}

Example 2.1 input PP

\begin{minted}{java}
class Main {
    public static void main(String[] args) {
        int a = new A().doX(1).doY("Hello").doZ(false);
    }
}
\end{minted}


Example 2.2 output stubs

\begin{minted}{java}
class A {
    UnknownType1 doX(int i) { return null; }
}
interface UnknownType1 {
    UnknownType2 doY(String s);
}
interface UnknownType2 {
    int doZ(boolean b);
}
\end{minted}

It is very apparent that it is possible for all the UnknownTypes to be
collapsed into one of the referenced types (for both PPs, the UnknownTypes
can be collapsed into class A), or for Example 2.2, the two UnknownTypes can
be collapsed into a single UnknownType. More work needs to be done to investigate
these.

\bibliography{javacpp}

\end{document}
